\section{\textbf{TUTORIAL}}
    \label{cha:tutorial}
\hypersetup{colorlinks=true, linkcolor=black, urlcolor=black}
    Não coloque \textbf{espaço} no nome dos arquivos, muito menos nas \textbf{referências dos nomes dos arquivos}, pois quando realizar as chamadas em texto torna\hyp{se} complicado tratar esse caractere.
    
    Padrões utilizados por todo o documento:
    \begin{itemize}
        \item Código identado para fácil leitura;
        \item Nome das pastas com todas as letras minúsculas;
        \item Nome dos arquivos (.tex | .bib) capitalizados;
        \item Separações dos nomes dos arquivos com - (hífem);
        \item Os rótulos dos elementos textuais possuem um prefixo sobre o tipo do item que representam, segue alguns exemplos.
        \begin{verbatim} 
            \label{cha:introducao}
            \label{sec:objetivos}
            \label{subsec:objetivo-geral}
            \label{subsubsec:assunto-xyz}
            \label{fig:nome-da-figura}
            \label{tab:nome-da-tabela}
        \end{verbatim} 
    \end{itemize}

\subsection{\textbf{Como criar referências}}
\label{sec:referencias}

\href{https://verbosus.com/bibtex-style-examples.html}{Neste Link} contém alguns exemplos das possíveis anotações que o \LaTeX{} aceita, basta seguir o padrão.

Segue um exemplo de como realizar o uso da citação, além disso, adianto que esse exemplo utiliza a anotação para citações na web.

Com o \LaTeX{}, é muito fácil escrever documentos profissionais~\cite{latex1995}. Este é um exemplo de referência indireta. No entanto, existem casos em que se faz necessário o uso de referência direta. Em \citeonline{latex1995}, é descrito como o \LaTeX{} é muito útil e simples de ser utilizado para gerar documentos profissionais.

Às vezes, a citação do autor pode ultrapassar 3 linhas, então faça uso desta ideia:
\begin{citacao}
    Segundo \citeonline{latex1995}
    \lipsum[59]
\end{citacao}

\begin{verbatim}
    \cite{latex1995} -> Referência indireta
    
    \citeonline{latex1995} -> Referência direta            
    Segundo \citeonline{latex1995}
    \begin{citacao}
        
    \end{citacao}
\end{verbatim}

Além disso, outros tipos de referências podem ser utilizados no LaTeX, como por exemplo:

\subsubsection{Citação de Artigo em Periódico:}
\begin{verbatim}
    \cite{artigo_periodico}
\end{verbatim}
\cite{artigo_periodico}

\subsubsection{Referência de Livro:}
\begin{verbatim}
    \cite{livro}
\end{verbatim}
\cite{livro}

\subsubsection{Citação de Capítulo de Livro:}
\begin{verbatim}
    \cite{capitulo_livro}
\end{verbatim}
\cite{capitulo_livro}

\subsubsection{Trabalho em Conferência:}
\begin{verbatim}
    \cite{trabalho_conferencia}
\end{verbatim}
\cite{trabalho_conferencia}

\subsubsection{Tese ou Dissertação:}
\begin{verbatim}
    \cite{tese}
\end{verbatim}
\cite{tese}

\subsubsection{Algorítmos:}
\begin{verbatim}
    \cite{algoritmo}
\end{verbatim}
\cite{algoritmo}

\subsubsection{Filme:}
\begin{verbatim}
    \cite{filme}
\end{verbatim}
\cite{filme}

\subsubsection{Vídeo:}
\begin{verbatim}
    \cite{video}
\end{verbatim}
\cite{video}

Estes exemplos demonstram como fazer citações e referências de diferentes tipos de documentos utilizando o LaTeX e o pacote `abntex2cite`, seguindo as normas ABNT.

Estes exemplos demonstram como fazer citações e referências de diferentes tipos de documentos utilizando o LaTeX e o pacote `abntex2cite`, seguindo as normas ABNT.

    \subsection{\textbf{Figuras, Tabelas, Quadros, Equação e Algoritmos} }       
    Os exemplos de inclusão e como referenciar e uma figura, tabela, quadro, equação e algoritmo podem ser observados na Figura~\ref{fig:figura}, Tabela~\ref{tab:tabela}, Quadro~\ref{tab:quadro}, Equação~\ref{eq:equacao}, Algoritmo~\ref{alg:algoritmo}.
    
    \begin{figure}[h]
        \centering
        \caption{Exemplo de Figura}
        \includegraphics{lib/Logoufrn.jpg}
        % Contém Fonte: 
        \source{O Autor (\imprimirdata)}
        \label{fig:figura}
    \end{figure}
    
    \begin{table}[h]
        \centering
    % \renewcommand{\tablename}{Quadro} % Texto a ser mostrado na lista
        \caption{Exemplo de Tabela}
        % c -> center, l -> left, r -> right e | -> linha vertical
        \begin{tabular}{c|c}
            \hline
            \textbf{Título} & \textbf{Título}  \\ \hline %\hline -> linha horizontal
            x1 & 23 \\ \hline 
        \end{tabular}
        \label{tab:tabela}
        \source{O Autor (\imprimirdata)}
    \end{table}

    
    \begin{quadro}[h]
        \centering
        \caption{Exemplo de Quadro}
        % c -> center, l -> left, r -> right e | -> linha vertical
        \begin{tabular}{|c|c|}
            \hline
            \textbf{Título} & \textbf{Título}  \\ \hline
            chave & valor em texto \\ \hline
        \end{tabular}
        \label{tab:quadro}
        \source{O Autor (\imprimirdata)}
    \end{quadro}
    
    \begin{equation}
        \label{eq:equacao}
        a^2 = b^2 + c^2
    \end{equation}
    
    \begin{algorithm}[H]
        \caption{Pseudo\hyp{Código} de Exemplo}
        \label{alg:algoritmo}
        \SetAlgoLined
        \begin{algorithmic}[1]
            \REQUIRE entender, estresse
            \ENSURE estresse $\leftarrow$ 0, entender $\leftarrow$ FALSE
            \REPEAT
            	\STATE leia o template com calma
            	\STATE tente entender
            	\IF{entender}
                    \STATE entender $\leftarrow$ TRUE
                \ELSE
                    \STATE estresse $\leftarrow$ estresse + 1
                \ENDIF
            \UNTIL{entender $=$ TRUE}
            \RETURN estresse
        \end{algorithmic}
    \end{algorithm}
    \begin{center}
        \vspace{-2em}
        \source{O Autor (\imprimirdata)}
    \end{center}
    
    \subsection{\textbf{Siglas}}
        \label{sec:siglas}
        
        No curso de \ac{bsi} da \ac{ufrn} é um curso da área de \ac{ti} na cidade de Caicó - RN.
\newpage
