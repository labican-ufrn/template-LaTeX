
%%%%%%%%%%%%%%%%%%%%%%%%%%%%%%%%%%%%%%
%%            Introdução            %%
%%%%%%%%%%%%%%%%%%%%%%%%%%%%%%%%%%%%%%

% Neste capítulo devem ser apresentados o contexto do seu trabalho e o problema que deseja abordar, bem como os objetivos que deseja alcançar e a justificativa para o estudo.

\section{\textbf{INTRODUÇÃO}}
    \label{sec:introdução}
    % Tópicos propostos, para remover algum destes tópicos
    %comente a linha referente ao tópico que deseja remover
    Neste capítulo devem ser apresentados o contexto do seu trabalho e o problema que deseja abordar, bem como os objetivos que deseja alcançar e a justificativa para o estudo.
    
    % Contextualização e problema
    
    
% Nesta seção devem ser introduzidos o ambiente (\textit{i.e.}Contexto) em que seu trabalho está inserido, além do problema que será abordado no seu trabalho.

\subsection{\textbf{Contextualização e Problema}}
    \label{subsec:contextualizacao-problema}

    Nesta seção devem ser introduzidos o ambiente (\textit{i.e.}Contexto) em que seu trabalho está inserido, além do problema que será abordado no seu trabalho.

    % Objetivos (geral e específicos)
    
% Esta seção contém os objetivos de sua pesquisa, contemplando o objetivo principal e as atividades para que este objetivo seja atingido.

\section{Objetivos}
    \label{sec:objetivos}
    
    Esta seção contém os objetivos de sua pesquisa, contemplando o objetivo principal e as atividades para que este objetivo seja atingido.

    \subsection{Objetivo Geral}
        \label{subsec:objetivo-geral}

    \subsection{Objetivos Específicos}
        \label{subsec:objetivos-especificos}
    % Delimitação do estudo
    
% Nesta seção deve ser apresentado o problema abordado no estudo de forma a especificá-lo dentro do contexto em que está inserido.

\subsection{\textbf{Delimitação do Estudo}}
    \label{sec:delimitacao-estudo}
    
    Nesta seção deve ser apresentado o problema abordado no estudo apresentando exatamente o que será considerado para o mesmo e deixando claro o que não será levado em conta. Considerando um trabalho que fará um estudo comparativo entre algoritmos a partir do desempenho em bases de dados.
    
    Neste trabalho serão observados apenas X técnicas, são elas, Técnica1, Técnica2, ..., TécnicaN, pois são técnicas utilizadas para classificação de dados em bancos de dados não relacionais. Dentro delas, Y parâmetros são abordados e Z não são pelos motivos AC/DC.
    
    % de forma a especificá-lo dentro do contexto em que está inserido.

    % Justificativa
    
% Esta seção trata os motivos pelos quais seu trabalho é relevante

\section{Justificativa}
    \label{sec:justificativa}
    
    Esta seção trata os motivos pelos quais seu trabalho é relevante, respondendo à estas perguntas:
    \begin{enumerate}
        \item O que a comunidade acadêmica irá ganhar com seu trabalho?;
        \item Qual a razão do seu trabalho ser desenvolvido?;
    \end{enumerate}
    
    % Apresentação do trabalho
    
% Esta seção trata a construção do documento, especificamente a disposição dos capítulos e uma introdução do que está sendo abordado nestes.

\subsection{\textbf{Apresentação do Trabalho}}
    \label{sec:apresentacao-trabalho}
    
    Esta seção trata a construção do documento, especificamente a disposição dos capítulos e uma introdução do que está sendo abordado nestes.


% Nesta seção devem ser introduzidos o ambiente (\textit{i.e.}Contexto) em que seu trabalho está inserido, além do problema que será abordado no seu trabalho.
\newpage
