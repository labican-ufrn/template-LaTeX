
%%%%%%%%%%%%%%%%%%%%%%%%%%%%%%%%%%%%%%
%%            Introdução            %%
%%%%%%%%%%%%%%%%%%%%%%%%%%%%%%%%%%%%%%

% Neste capítulo devem ser apresentados o contexto do seu trabalho e o problema que deseja abordar, bem como os objetivos que deseja alcançar e a justificativa para o estudo.

\chapter{Introdução}
    \label{cha:intro}

    Neste capítulo devem ser apresentados o contexto do seu trabalho e o problema que deseja abordar, bem como os objetivos que deseja alcançar e a justificativa para o estudo.
    
    Os exemplos de inclusão e como referenciar e uma figura, tabela, quadro e algoritmo podem ser observados na Figura~\ref{fig:figura}, Tabela~\ref{tab:tabela}, Quadro~\ref{tab:quadro}
    
    \begin{figure}[h]
        \centering
        \caption{Exemplo de Figura}
        \includegraphics{lib/logoufrn.jpg}
        % Contém Fonte: 
        \source{O Autor (\imprimirdata)}
        \label{fig:figura}
    \end{figure}
    
    \begin{table}[h]
        \centering
        \caption{Exemplo de Tabela}
        % c -> center, l -> left, r -> right e | -> linha vertical
        \begin{tabular}{|c|c|}
            \hline
            \textbf{Título} & \textbf{Título}  \\ \hline %\hline -> linha horizontal
            x1 & 23 \\ \hline 
        \end{tabular}
        \label{tab:tabela}
        \source{O Autor (\imprimirdata)}
    \end{table}
    
    \begin{quadro}[h]
        \centering
        \caption{Exemplo de Quadro}
        % c -> center, l -> left, r -> right e | -> linha vertical
        \begin{tabular}{c|c}
            \textbf{Título} & \textbf{Título}  \\ \hline
            chave & valor em texto
        \end{tabular}
        \label{tab:quadro}
        \source{O Autor (\imprimirdata)}
    \end{quadro}
    
    % Tem que ajustar para cada variação, faça isso ao fim do trabalho, coloque no título do Algoritmo ou deixa da seguinte forma com esse H.
    \begin{algorithm}[H]
        \caption{Pseudo\hyp{Código} de Exemplo}
        \label{alg:algoritmo}
        \SetAlgoLined
        \begin{algorithmic}[1]
            \REQUIRE entender, estresse
            \ENSURE estresse $\leftarrow$ 0, entender $\leftarrow$ FALSE
            \REPEAT
            	\STATE leia o template com calma
            	\STATE tente entender
            	\IF{entender}
                    \STATE entender $\leftarrow$ TRUE
                \ELSE
                    \STATE estresse $\leftarrow$ estresse + 1
                \ENDIF
            \UNTIL{entender $\neq$ TRUE}
            \RETURN estresse
        \end{algorithmic}
    \end{algorithm}
    \begin{center}
        \vspace{-2em}
        \source{O Autor (\imprimirdata)}
    \end{center}
    
    
    % Tópicos propostos, para remover algum destes tópicos
    %comente a linha referente ao tópico que deseja remover
    
    % Contextualização e problema
    \section{Contextualização e Problema}
    \label{subsec:contextualizacao-problema}
    
    A \index{cor}cor do meu carro é a cor Amarela
    % Objetivos (geral e específicos)
    
% Esta seção contém os objetivos de sua pesquisa, contemplando o objetivo principal e as atividades para que este objetivo seja atingido.

\subsection{\textbf{Objetivos}}
    \label{sec:objetivos}
    
    Esta seção contém os objetivos de sua pesquisa, contemplando o objetivo principal e as atividades para que este objetivo seja atingido.

    % Delimitação do estudo
    \section{Delimitação do Estudo}
    \label{sec:delimitacao-estudo}
    % Justificativa
    
% Esta seção trata os motivos pelos quais seu trabalho é relevante

\subsection{\textbf{Justificativa}}
    \label{sec:justificativa}
    
    Esta seção trata os motivos pelos quais seu trabalho é relevante, respondendo à estas perguntas:
    \begin{enumerate}
        \item O que a comunidade acadêmica irá ganhar com seu trabalho?;
        \item Qual a razão do seu trabalho ser desenvolvido?;
    \end{enumerate}
    
    % Apresentação do trabalho
    
% Esta seção trata a construção do documento, especificamente a disposição dos capítulos e uma introdução do que está sendo abordado nestes.

\subsection{\textbf{Apresentação do Trabalho}}
    \label{sec:apresentacao-trabalho}
    
    Esta seção trata a construção do documento, especificamente a disposição dos capítulos e uma introdução do que está sendo abordado nestes.


