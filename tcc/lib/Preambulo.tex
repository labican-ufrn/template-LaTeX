%%%%%%%%%%%%%%%%%%%%%%%%%%%%%%%%%%%%%%%%%%%%%%%%%%
%%                                              %%
%%          Importação de pacotes               %%
%%                                              %%
%%%%%%%%%%%%%%%%%%%%%%%%%%%%%%%%%%%%%%%%%%%%%%%%%%

% Acrônimos (Siglas e Abreviaturas)
\usepackage{acronym}

\hypersetup{hidelinks} % Isso desativa a coloração dos links

% Pacote para o desenvolvimento de algoritmos e codificação utf8
\usepackage[portuguese,lined,boxed,ruled]{algorithm2e}
\usepackage{algorithmic}

% Fontes e símbolos matemáticos
\usepackage{amsfonts, amsmath, amssymb}

\usepackage[brazil]{babel}

\usepackage{booktabs}


% Manutenção das legendas em imagens (fonte pequena, 10pt)
\usepackage[font=small]{caption}

% Manutenção da marcação em listas (enumerator)
\usepackage{enumitem}

% Codificação da fonte em 8 bits
\usepackage[T1]{fontenc}

% Pacote para formatação de títulos
\usepackage{titlesec}

% Configuração da fonte para Arial (Helvetica)
\renewcommand{\rmdefault}{phv}
\renewcommand{\sfdefault}{phv}

\titleformat{\section}
  {\bfseries\normalsize} % Define a fonte para Arial (Helvetica) e tamanho 12 para as seções
  {\thesection}
  {1em}
  {}

\titleformat{\subsection}
  {\normalsize\bfseries} % Define negrito e tamanho 12 para as subseções
  {\thesubsection}
  {1em}
  {}

\titleformat{\subsubsection}
  {\normalsize} % Define a fonte para Arial (Helvetica) e tamanho 12 para as subsubseções sem negrito
  {\thesubsubsection}
  {1em}
  {}

% Inserir figuras
\usepackage{graphicx}

% Dimensões do documento
\usepackage[left=3.0cm, right=2.0cm, top=3.0cm, bottom=2.0cm]{geometry}

% Glossário
\usepackage[nonumberlist,style=index]{glossaries}

% Hifenização das palavras 
\usepackage{hyphenat}

% Índice
\usepackage{imakeidx}

% Identar do primeiro parágrafo de cada seção
\usepackage{indentfirst}

% Acentuação direta
\usepackage[utf8]{inputenc}

% Gerador de Texto
\usepackage{lipsum}

% Para melhorias de justificação
\usepackage{microtype}

% Mescla de células em tabelas
\usepackage{multirow}

\usepackage{placeins}

% Rotacionar elementos
\usepackage{rotating}

% Manutenção do espaçamento entre linhas
\usepackage{setspace}

% Cria rodapé em tabelas 
\usepackage{threeparttable}

% Texto em Arial (Helvetica)
\renewcommand{\rmdefault}{phv}
\renewcommand{\sfdefault}{phv}

% Pacotes com Hierarquia
\usepackage[brazilian, hyperpageref]{backref} % Paginas com as citações na bibl

% Referências
\usepackage[alf, bibjustif, abnt-etal-list=0, abnt-etal-text=it]{abntex2cite}

% Redefine o formato do título das referências
\renewcommand{\bibsection}{%
  \clearpage
  \phantomsection
  \addcontentsline{toc}{section}{\protect\numberline{}REFERÊNCIAS} % Adiciona o título ao sumário
  {\centering\bfseries\fontsize{12}{14}\selectfont REFERÊNCIAS\par\nobreak} % Título das referências
}

% Configurações do pacote hyperref
\usepackage{hyperref}

% Configurações do pacote backref
% Usado sem a opção hyperpageref de backref
\renewcommand{\backrefpagesname}{}
% Texto padrão antes do número das páginas
\renewcommand{\backref}{}
% Define os textos da citação
\renewcommand*{\backrefalt}[4]{}

% Configurações das tabelas
