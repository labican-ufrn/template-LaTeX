%%%%%%%%%%%%%%%%%%%%%%%%%%%%%%%%%%%%%%%%%%%%%%%%%%
%%                                              %%
%%        Comandos usados no documento          %%
%%                                              %%
%%%%%%%%%%%%%%%%%%%%%%%%%%%%%%%%%%%%%%%%%%%%%%%%%%
% Capa e Folha de Rosto e Folha de Aprovação
\renewcommand{\imprimircapa}{
%%%%%%%%%%%%%%%%%%%%%%%%%%%%%%%%%%%%%%%%%%%%%%%%%%
%%                                              %%
%%              Início da Capa                  %%
%%                                              %%
%%%%%%%%%%%%%%%%%%%%%%%%%%%%%%%%%%%%%%%%%%%%%%%%%%

\thispagestyle{empty}

\begin{figure}
    \centering
    \includegraphics[]{lib/logoufrn.jpg}
\end{figure}

\begin{center}
    \Large{\imprimirinstituicao}
    
    \vspace{\stretch{2}}
    
    \large{\textsc{\imprimirtitulo}}
    
    \vspace{\stretch{2}}
    
    \textsc{\large{\textbf{\imprimirautor}}}
    
    \vspace{\stretch{3}}
    
    \large{\imprimirlocal} \\
    \large{\imprimirdata}
\end{center}

\newpage
%%%%%%%%%%%%%%%%%%%%%%%%%%%%%%%%%%%%%%%%%%%%%%%%%%
%%                                              %%
%%                  Fim da Capa                 %%
%%                                              %%
%%%%%%%%%%%%%%%%%%%%%%%%%%%%%%%%%%%%%%%%%%%%%%%%%%}
\renewcommand{\imprimirfolhaderosto}{
%%%%%%%%%%%%%%%%%%%%%%%%%%%%%%%%%%%%%%%%%%%%%%%%%%
%%                                              %%
%%          Início da Folha de Rosto            %%
%%                                              %%
%%%%%%%%%%%%%%%%%%%%%%%%%%%%%%%%%%%%%%%%%%%%%%%%%%

\thispagestyle{empty}

\begin{center}
    \large{\imprimirautor}\\[3cm] 
    \large \textbf{\imprimirtitulo}\\[3cm] 
    
    \hfill
    \begin{minipage}{.5\linewidth}
        \small\imprimirpreambulo.
        \\ [0.6cm]
        \small Orientador: \imprimirorientador.
        \\
        %\small Co-Orientador: \imprimircoorientador.
    \end{minipage}
    \\ [6.3cm]

    \large{\imprimirlocal}\\
    \large{\imprimirdata}
\end{center}

\newpage
%%%%%%%%%%%%%%%%%%%%%%%%%%%%%%%%%%%%%%%%%%%%%%%%%%
%%                                              %%
%%            Fim da Folha de Rosto             %%
%%                                              %%
%%%%%%%%%%%%%%%%%%%%%%%%%%%%%%%%%%%%%%%%%%%%%%%%%%
}

% subtítulo
\providecommand{\imprimirsubtitulo}{}
\newcommand{\subtitulo}[1]{\renewcommand{\imprimirsubtitulo}{#1}}

% criar fonte em imagens, tabelas e quadros
\newcommand{\source}[1]{\legend{\textbf{Fonte:} {#1}}}    

% texto em arial
%\renewcommand{\sfdefault}{phv}
%\renewcommand{\rmdefault}{phv}

% texto em Times
%\renewcommand{\sfdefault}{ptm}
%\renewcommand{\rmdefault}{ptm}

%%%%%%%%%%%%%%%%%%%%%%%%%%%%%%%%%%%%%%%%%%%%%%%%
%%          Informações do documento          %%
%%%%%%%%%%%%%%%%%%%%%%%%%%%%%%%%%%%%%%%%%%%%%%%%


% Preencha com os seus dados
\titulo{TITULO DO TRABALHO: SUBTÍTULO}
\tipotrabalho{Monografia (bacharel em Sistema de Informação) }
\autor{AUTOR DO TRABALHO}
\orientador{Orientador do Trabalho}     % nível acadêmico + nome
\local{CIDADE - RN}
\data{\the\year}
\instituicao{
    UNIVERSIDADE FEDERAL DO RIO GRANDE DO NORTE\par
    CENTRO DE ENSINO SUPERIOR DO SERIDÓ \par
    CURSO DE ... \par
}

% Rótulo do orientador
\renewcommand{\imprimirorientadorRotulo}{Orientador(a): }
\renewcommand{\imprimircoorientadorRotulo}{Co-orientador(a): }

\preambulo{Trabalho de conclusão de curso apresentado ao curso de graduação em (Nome do Curso), como parte dos requisitos para obtenção do título de (Título/ em (Curso) Federal do Rio Grande do Norte.}

%%%%%%%%%%%%%%%%%%%%%%%%%%%%%%%%%%%%%%%%%%%%%%%%%%
%%                                              %%
%%        Comandos usados no documento          %%
%%                                              %%
%%%%%%%%%%%%%%%%%%%%%%%%%%%%%%%%%%%%%%%%%%%%%%%%%%

%%%%%%%%%%%%%%%%%%%%%%%%%%%%%%%%%%
%%                              %%
%%            Tabelas           %%
%%                              %%
%%%%%%%%%%%%%%%%%%%%%%%%%%%%%%%%%%
\newcommand{\quadroname}{Quadro}

\newfloat[chapter]{quadro}{loq}{\quadroname}
\newlistof{listofquadros}{loq}{\listofquadrosname}
\newlistentry{quadro}{loq}{0}

% configurações para atender às regras da ABNT
\setfloatadjustment{quadro}{\centering}
\counterwithout{quadro}{chapter}
\renewcommand{\cftquadroname}{\quadroname\space} 
\renewcommand*{\cftquadroaftersnum}{\hfill--\hfill}

\setfloatlocations{quadro}{hbtp}


%%%%%%%%%%%%%%%%%%%%%%%%%%%%%%%%%%%%%%%%%%%%%%%%%%
%%                                              %%
%%   Configurações de aparência do PDF final    %%
%%                                              %%
%%%%%%%%%%%%%%%%%%%%%%%%%%%%%%%%%%%%%%%%%%%%%%%%%%

%%%%%%%%%%%%%%%%%%%%%%%%%%%%%%%%%%
%%                              %%
%%      Informações do PDF      %%
%%                              %%
%%%%%%%%%%%%%%%%%%%%%%%%%%%%%%%%%%
\makeatletter
\hypersetup{
    %  	pagebackref=true,
		pdftitle={\@title}, 
		pdfauthor={\@author},
        pdfsubject={\imprimirpreambulo},
        pdfcreator={LaTeX with abnTeX2},
		pdfkeywords={abnt}{latex}{abntex}{abntex2}{trabalho acadêmico}, 
    %   false: boxed links; true: colored links
		colorlinks=true,
    % 	color of internal links
        linkcolor=black,
    %   color of links to bibliography
        citecolor=black,
    %   color of file links
        filecolor=magenta,
		urlcolor=blue,
		bookmarksdepth=4
}
\makeatother

%%%%%%%%%%%%%%%%%%%%%%%%%%%%%%%%%%
%%                              %%
%%         Indentação           %%
%%                              %%
%%%%%%%%%%%%%%%%%%%%%%%%%%%%%%%%%%

% O tamanho do parágrafo é dado por:
\setlength{\parindent}{1.5cm}

% Controle do espaçamento entre um parágrafo e outro:
\setlength{\parskip}{0.2cm}  % tente também \onelineskip


%%%%%%%%%%%%%%%%%%%%%%%%%%%%%%%%%%
%%                              %%
%%         Algoritmos           %%
%%                              %%
%%%%%%%%%%%%%%%%%%%%%%%%%%%%%%%%%%

% Algoritmos
% \floatname{algorithm}{Algoritmo}%Algoritmo
% \renewcommand{\listalgorithmname}{LISTA DE ALGORITMOS}

\renewcommand{\algorithmicindent}{3.0em}
\renewcommand{\algorithmicrequire}{\textbf{entrada}}
\renewcommand{\algorithmicensure}{\textbf{garanta}}
\renewcommand{\algorithmicreturn}{\textbf{retorne}}
\renewcommand{\algorithmicor}{\textbf{ou}}
\renewcommand{\algorithmicand}{\textbf{e}}
\renewcommand{\algorithmicend}{\textbf{fim}}
\renewcommand{\algorithmicif}{\textbf{se}}
\renewcommand{\algorithmicelse}{\textbf{sen\~ao}}
\renewcommand{\algorithmicthen}{\textbf{ent\~ao}}
\renewcommand{\algorithmicfor}{\textbf{para}}
\renewcommand{\algorithmicforall}{\textbf{para cada}}
\renewcommand{\algorithmicrepeat}{\textbf{repita}}
\renewcommand{\algorithmicuntil}{\textbf{até que}}
\renewcommand{\algorithmicwhile}{\textbf{enquanto}}
\renewcommand{\algorithmicdo}{\textbf{fa\c{c}a}}

